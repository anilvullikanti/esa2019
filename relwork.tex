\vspace{-0.1in}
\section{Related Work}
\label{sec:related}
The works by \cite{kkt-2003}, \cite{v011a004}, were the first to formulate the problem of influence
spreading as a discrete
optimization problem. They consider two main diffusion models: \textit{independent cascade} (which we adopt in this paper) and
\textit{linear threshold}. Works in these models
include two main optimization problems: maximizing the expected influence with constraints on the
seed set (usually size), and minimizing the seed set (according to some measurement) to achieve a target expected
influence. These problems are at least NP-hard, \cite{v011a004}. In \cite{chen2010scalable}, it
was shown that computing the expected influence is \#P-hard. These results set a theoretical limit
on what can be done to attack the problems.

The approach proposed by \cite{kkt-2003}, using submodular
set function and greedy heuristic, gives a $(1-1/e-\epsilon)$-approximation to the maximizing the expected
influence problem. The error $\epsilon$ is due to Monte Carlo estimation of the expected
influence, which is the main issue in practice, where large real world graphs discourage
excessive sampling. \cite{borgs2014maximizing} propose an algorithm with
nearly optimal theoretical runtime of $O((m+n)k\epsilon^{-2}\log n)$ while retaining the same
approximation guarantee. The technique is to sample \textit{reversed} influence, which is adopted
and improved upon in other works, e.g., \cite{tang2014influence}, \cite{tang2015influence},
\cite{nguyen2016stop}. While reversed influence sampling has a theoretical guarantee for the
expectation problem, it is not extendable to our probabilistic \mdelta.
Another approach, proposed by \cite{lucier:kdd15}, is to
estimate expected influence as a Riemann sum, using $O(n \epsilon^2 \text{polylog}(n))$ samples
which can be implemented in  parallel by using MapReduce.

Other models are also studied, for example, fixed threshold models \cite{chen2009approximability},
\cite{goldberg2013diffusion}, time-restricted diffusion model
\cite{goyal2013minimizing}, \cite{chen2012time}, \cite{dinh2014cost},
continuous-time diffusion model
\cite{du2013scalable}, and diffusion in dynamic network \cite{tong2017adaptive}.

The work closest to ours is presented by \cite{zhang:kdd14}. They
propose to measure influence with a probabilistic guarantee; their  goal is to find a minimum-sized seed set   that achieves a given
target influence with a specified probability. They show that their problem is \#P-hard, and give an
approximation algorithm. The algorithm uses expected influence as the criteria (which requires high number of samples) to select
members of the seed set, and then verifies the probability condition.
They show that the size of the output seed set, compared to the
optimal one, incurs both a multiplicative error of $(\ln n + O(1))$ and an additive error
of $O(\sqrt{n})$, under the assumption that the standard deviation of the influence is
$O(\sqrt{n})$. In contrast, our goal is different ---
we want to maximize influence with probabilistic guarantee, with a constraint on the seed set
size. Our multi-criteria approximation does not incurs additive error, and does not relies
on assumption about the distribution.
