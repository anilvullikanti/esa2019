\vspace{-0.2in}
\section{Introduction}
\label{sec:intro}

A large number of phenomena, e.g., the spread of influence, fads and ideologies on social
networks, can be modeled as a diffusion process on a graph; see, e.g., \cite{ek-2010,v011a004}.
From a given seed set $S$, let the influence spreads by some probabilistic models
(see \ref{sec:model} and \ref{sec:related}), and take the size of the final influenced
set, denoted $I(S)$, as the \textit{influence} of $S$. One optimization problem, \textit{influence
maximization}, is stated as follows: choose a seed set $S$ of size $k$, so
that the expectation, $E[I(S)]$, is maximized---this is referred to as the
 \infmax{} problem.
The seminal work by Kempe, Kleinberg and Tardos \cite{kkt-2003} was the first to give a constant
factor approximation to the \infmax{} problem.
There has been a lot of work on many variants of influence maximization for several diffusion
models; see, e.g., \cite{v011a004,ek-2010} for details.

However, there are instances, in which the expected influence set is large, but
the variance is also high, which is not desirable.
% As a result, the optimum seed set $S$, which maximizes $E[I(S)]$, could have the
% unsatisfactory property that  $\Pr[I(S)\geq \alpha]$ (i.e., the probability that the number of
% influenced nodes is at least $\alpha$) is not very high (cf. Lemma \ref{lem:example}).
A common way to understand the variance in a random variable is by examining its quantiles.
Motivated by this, we focus on the problem of finding a seed set $S$ of size $k$ such that the
$\delta$ - quantile value (for a given $\delta$, $0 < \delta \leq 1$) of $I(S)$ is maximized ---
we denote this by \maxinfdelta $= \max\{\alpha| \Pr[I(S)\geq\alpha] \geq\delta\}$. The focus
of this paper is to find $S$ that maximizes \maxinfdelta; we refer to this as the $\infprob{}$
problem.

The first prior work on influence maximization with probabilistic guarantees was by
\cite{zhang:kdd14}: they introduce the \emph{Seed minimization with probabilistic coverage guarantee} (SM-PCG) problem: find the smallest seed set $S$ that ensures that $\Pr[I(S)\geq\eta]\geq P$, where $\eta$ and $P$ are parameters. They also give an additive approximation to the minimum seed set size needed for a given $\eta, P$, and show that the solutions achieve similar expected influence, but with the additional guarantee on the probability.
There are two limitations of their work. First, the additive approximation is $O(\sqrt{n})$,
which can be quite large. In contrast, influence maximization has typically been studied for
bounded seed set sizes, which, ideally, should be small. Second, the achievable influence $\eta$
is not known a priori, and so the algorithm would have to be run for multiple $\eta$ values to
understand a tradeoff.

In this paper, we study the structure of \infprob{} problem, and develop efficient approximation
algorithms for it. Our contributions are the following.\\
%\begin{itemize}
%\item
\noindent {\bf 1.}  We formalize a more natural notion of influence maximization with probabilistic
guarantees as the quantile value. We also study the structure of solutions to \infprob{}, and what
 effect the probabilistic guarantees have. These were not examined in \cite{zhang:kdd14}.

\noindent {\bf 2.}
We develop a multi-criteria approximation algorithm, \multicritalgo{}, with \emph{multiplicative
factor} guarantees (instead of both multiplicative and additive guarantee, in \cite{zhang:kdd14}).
We also design and
analyze an efficient sampling method, \infprobheu{}, to estimate $M_{\delta}(I(S))$ for any given
 seed set $S$.

A main technical novelty of our work is the analysis of our multi-criteria approximation algorithm.
We use the \emph{Sample Average Approximation} (SAA) technique from stochastic optimization
(see, e.g., \cite{swamy:sigact06}), which involves constructing samples $G_1,\ldots,G_N$ of
the graph, as per the IC model. Since it can be shown that $\mdelta$ is not submodular (cf. Section \ref{sec:heu}), we define a
different type of submodular function $F_{\lambda}(S)$, using the \emph{saturation} technique of \cite{krause:jmlr08}. 
% which thresholds the influence of $S$
% in each $G_i$ to $\lambda$, and takes an average. By adapting the techniques of \cite{krause:jmlr08},
We show that finding a minimum cost set $S$ that ensures $F_{\lambda}(S)\geq \delta\lambda$ is sufficient to give a multi-criteria approximation---this
problem is a variant of the standard submodular cover problem. However, the problem does not
satisfy an important technical requirement in the submodular cover problem, and we need to
modify the analysis of \cite{wolsey:combinatorica82} to account for this difference.

\noindent {\bf 3.}
Finally, we evaluate our methods on several datasets.  Not surprisingly, the solutions to
$\mdelta$ computed using \multicritalgo{} and \infprobheu{}, have similar or higher (up to 10\% in some cases) quantile values than the \infmax{} solution. More importantly, we observe that the running time of both \multicritalgo{} and \infprobheu{} is significantly faster (about 10 times faster) compared to the standard algorithm   that implements \infmax{} (due to Kempe et al \cite{v011a004}).  In fact,  we also observe that our solutions have similar or better \emph{expected influence value}, than the \infmax{} solution (though this objective was not being optimized); additionally, we get bounds on the probability.
In particular, we find that for many instances with low activation probabilities, the seed
set output by \multicritalgo{} yields even better (by up to 10\%) expected influence value than
the \infmax{} solution, computed using the approximation algorithm of \cite{kkt-2003} which
gives a constant-factor approximation to the optimal expected influence.
%\end{itemize}
% We show that $\mdelta$ is not a submodular function of $S$.

All omitted proofs and some additional details are  presented in the full version
of the paper \url{www.dropbox.com/sh/r6i10b6s18gduap/AAD-4ZUScaq8_-RBeDtCsp2oa?dl=0}.
