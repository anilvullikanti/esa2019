
\documentclass[a4paper,UKenglish,cleveref, autoref]{lipics-v2019}
%This is a template for producing LIPIcs articles.
%See lipics-manual.pdf for further information.
%for A4 paper format use option "a4paper", for US-letter use option "letterpaper"
%for british hyphenation rules use option "UKenglish", for american hyphenation rules use option "USenglish"
%for section-numbered lemmas etc., use "numberwithinsect"
%for enabling cleveref support, use "cleveref"
%for enabling cleveref support, use "autoref"


%\graphicspath{{./graphics/}}%helpful if your graphic files are in another directory

\bibliographystyle{plainurl}% the mandatory bibstyle

\title{Dummy title} %TODO Please add

\titlerunning{Dummy short title}%optional, please use if title is longer than one line

\author{Author Name}{Dummy University Computer Science Department,
USA \and \url{http://www.myhomepage.edu} }
{johnqpublic@dummyuni.org}{}
{}
%TODO mandatory, please use full name; only 1 author per \author macro;
% first two parameters are mandatory, other parameters can be empty.
% Please provide at least the name of the affiliation and the country. The full address is optional

\authorrunning{J.\,Q. Public and J.\,R. Public}
%TODO mandatory. First: Use abbreviated first/middle names.
%Second (only in severe cases): Use first author plus 'et al.'

\Copyright{John Q. Public and Joan R. Public}
%TODO mandatory, please use full first names.
% LIPIcs license is "CC-BY";  http://creativecommons.org/licenses/by/3.0/

\ccsdesc[100]{Theory of computation~Numeric approximation algorithms}
%TODO mandatory: Please choose ACM 2012 classifications from https://dl.acm.org/ccs/ccs_flat.cfm

\keywords{Graph data mining, Influence maximization, Probabilistic guarantee}
%TODO mandatory; please add comma-separated list of keywords

\category{}%optional, e.g. invited paper

\relatedversion{}%optional, e.g. full version hosted on arXiv, HAL, or other respository/website
%\relatedversion{A full version of the paper is available at \url{...}.}

\supplement{}%optional, e.g. related research data, source code, ... hosted on a repository like zenodo, figshare, GitHub, ...

%\funding{(Optional) general funding statement \dots}%optional, to capture a funding statement, which applies to all authors. Please enter author specific funding statements as fifth argument of the \author macro.

% \acknowledgements{I want to thank \dots}%optional

\nolinenumbers %uncomment to disable line numbering

\hideLIPIcs  %uncomment to remove references to LIPIcs series (logo, DOI, ...),
%  e.g. when preparing a pre-final version to be uploaded to arXiv or another public repository

%Editor-only macros:: begin (do not touch as author)%%%%%%%%%%%%%%%%%%%%%%%%%%%%%%%%%%
% \EventEditors{John Q. Open and Joan R. Access}
% \EventNoEds{2}
% \EventLongTitle{42nd Conference on Very Important Topics (CVIT 2016)}
% \EventShortTitle{CVIT 2016}
% \EventAcronym{CVIT}
% \EventYear{2016}
% \EventDate{December 24--27, 2016}
% \EventLocation{Little Whinging, United Kingdom}
% \EventLogo{}
% \SeriesVolume{42}
% \ArticleNo{23}
%%%%%%%%%%%%%%%%%%%%%%%%%%%%%%%%%%%%%%%%%%%%%%%%%%%%%%

\usepackage{mathtools}
\usepackage{booktabs}
\usepackage{algorithm}
\usepackage[noend]{algpseudocode}
\usepackage{import}
\usepackage{pgfplots}
\pgfplotsset{compat=1.14}
\usepgfplotslibrary{groupplots}
\pgfplotsset{
  every axis/.append style={
    enlargelimits=false,
    scale only axis,
    scaled x ticks=false,
    scaled y ticks=false,
    legend style={fill=none, draw=none},
  }
}

\newcommand{\mycomment}[2][Comment]{\textcolor{blue} {\textsc{#1:} \emph{#2}}}

\newcommand{\infmax}{\textsc{MaxExpInf}}
\newcommand{\infprob}{\textsc{MaxProbInf}}
\newcommand{\infprobheu}{\textsc{ProbInf-Heu}}
\newcommand{\bicrit}{\textsc{Bicriteria}}
\newcommand{\bicritalgo}{\textsc{BiCritMdelta}}
\newcommand{\multicritalgo}{\textsc{MultiCritMdelta}}
\newcommand{\mdelta}{\mbox{$M_{\delta}(I(S))$}}
\newcommand{\expinf}{\textsc{ExpInf}}
\newcommand{\red}[1]{\textcolor{red}{#1}}
\newcommand{\maxinfdelta}{$M_{\delta}(I(S))$}
\newcommand{\iinf}{\textsc{Inf}}
\DeclareMathOperator*{\argmin}{argmin}
\DeclareMathOperator*{\argmax}{argmax}
\DeclareMathOperator{\E}{E}
\DeclareMathOperator{\Var}{Var}
\DeclareMathOperator{\Cov}{Cov}

\DeclarePairedDelimiter{\abs}{\lvert}{\rvert}

% \newtheorem{theorem}{Theorem}
% \newtheorem{lemma}[theorem]{Lemma}
% \newtheorem{definition}{Definition}
\newtheorem{fact}[definition]{Fact}
\newtheorem{observation}[definition]{Observation}

\begin{document}

\maketitle

%TODO mandatory: add short abstract of the document
\begin{abstract}
The standard Influence maximization problem involves choosing a seed set of a given size,
which maximizes the expected influence. However, such solutions might have a significant
probability of achieving low influence, which might not be suitable in many applications.
In this paper, we consider a different approach: find a seed set which maximizes the influence set
size, which can be achieved with a given probability. We show that this objective is not
submodular, and design two algorithms for this problem, one of which gives rigorous approximation
bounds. We evaluate our algorithms on multiple datasets, and show that they have similar or better
performance as the ones optimizing the expected influence, but with additional guarantees on the
probability.
\end{abstract}

\vspace{-0.2in}
\section{Introduction}
\label{sec:intro}

A large number of phenomena, e.g., the spread of influence, fads and ideologies on social
networks, can be modeled as a diffusion process on a graph; see, e.g., \cite{ek-2010,v011a004}.
From a given seed set $S$, let the influence spreads by some probabilistic models
(see \ref{sec:model} and \ref{sec:related}), and take the size of the final influenced
set, denoted $I(S)$, as the \textit{influence} of $S$. One optimization problem, \textit{influence
maximization}, is stated as follows: choose a seed set $S$ of size $k$, so
that the expectation, $E[I(S)]$, is maximized---this is referred to as the
 \infmax{} problem.
The seminal work by Kempe, Kleinberg and Tardos \cite{kkt-2003} was the first to give a constant
factor approximation to the \infmax{} problem.
There has been a lot of work on many variants of influence maximization for several diffusion
models; see, e.g., \cite{v011a004,ek-2010} for details.

However, there are instances, in which the expected influence set is large, but
the variance is also high, which is not desirable.
% As a result, the optimum seed set $S$, which maximizes $E[I(S)]$, could have the
% unsatisfactory property that  $\Pr[I(S)\geq \alpha]$ (i.e., the probability that the number of
% influenced nodes is at least $\alpha$) is not very high (cf. Lemma \ref{lem:example}).
A common way to understand the variance in a random variable is by examining its quantiles.
Motivated by this, we focus on the problem of finding a seed set $S$ of size $k$ such that the
$\delta$ - quantile value (for a given $\delta$, $0 < \delta \leq 1$) of $I(S)$ is maximized ---
we denote this by \maxinfdelta $= \max\{\alpha| \Pr[I(S)\geq\alpha] \geq\delta\}$. The focus
of this paper is to find $S$ that maximizes \maxinfdelta; we refer to this as the $\infprob{}$
problem.

The first prior work on influence maximization with probabilistic guarantees was by
\cite{zhang:kdd14}: they introduce the \emph{Seed minimization with probabilistic coverage guarantee} (SM-PCG) problem: find the smallest seed set $S$ that ensures that $\Pr[I(S)\geq\eta]\geq P$, where $\eta$ and $P$ are parameters. They also give an additive approximation to the minimum seed set size needed for a given $\eta, P$, and show that the solutions achieve similar expected influence, but with the additional guarantee on the probability.
There are two limitations of their work. First, the additive approximation is $O(\sqrt{n})$,
which can be quite large. In contrast, influence maximization has typically been studied for
bounded seed set sizes, which, ideally, should be small. Second, the achievable influence $\eta$
is not known a priori, and so the algorithm would have to be run for multiple $\eta$ values to
understand a tradeoff.

In this paper, we study the structure of \infprob{} problem, and develop efficient approximation
algorithms for it. Our contributions are the following.\\
%\begin{itemize}
%\item
\noindent {\bf 1.}  We formalize a more natural notion of influence maximization with probabilistic
guarantees as the quantile value. We also study the structure of solutions to \infprob{}, and what
 effect the probabilistic guarantees have. These were not examined in \cite{zhang:kdd14}.

\noindent {\bf 2.}
We develop a multi-criteria approximation algorithm, \multicritalgo{}, with \emph{multiplicative
factor} guarantees (instead of both multiplicative and additive guarantee, in \cite{zhang:kdd14}).
We also design and
analyze an efficient sampling method, \infprobheu{}, to estimate $M_{\delta}(I(S))$ for any given
 seed set $S$.

A main technical novelty of our work is the analysis of our multi-criteria approximation algorithm.
We use the \emph{Sample Average Approximation} (SAA) technique from stochastic optimization
(see, e.g., \cite{swamy:sigact06}), which involves constructing samples $G_1,\ldots,G_N$ of
the graph, as per the IC model. Since it can be shown that $\mdelta$ is not submodular (cf. Section \ref{sec:heu}), we define a
different type of submodular function $F_{\lambda}(S)$, using the \emph{saturation} technique of \cite{krause:jmlr08}. 
% which thresholds the influence of $S$
% in each $G_i$ to $\lambda$, and takes an average. By adapting the techniques of \cite{krause:jmlr08},
We show that finding a minimum cost set $S$ that ensures $F_{\lambda}(S)\geq \delta\lambda$ is sufficient to give a multi-criteria approximation---this
problem is a variant of the standard submodular cover problem. However, the problem does not
satisfy an important technical requirement in the submodular cover problem, and we need to
modify the analysis of \cite{wolsey:combinatorica82} to account for this difference.

\noindent {\bf 3.}
Finally, we evaluate our methods on several datasets.  Not surprisingly, the solutions to
$\mdelta$ computed using \multicritalgo{} and \infprobheu{}, have similar or higher (up to 10\% in some cases) quantile values than the \infmax{} solution. More importantly, we observe that the running time of both \multicritalgo{} and \infprobheu{} is significantly faster (about 10 times faster) compared to the standard algorithm   that implements \infmax{} (due to Kempe et al \cite{v011a004}).  In fact,  we also observe that our solutions have similar or better \emph{expected influence value}, than the \infmax{} solution (though this objective was not being optimized); additionally, we get bounds on the probability.
In particular, we find that for many instances with low activation probabilities, the seed
set output by \multicritalgo{} yields even better (by up to 10\%) expected influence value than
the \infmax{} solution, computed using the approximation algorithm of \cite{kkt-2003} which
gives a constant-factor approximation to the optimal expected influence.
%\end{itemize}
% We show that $\mdelta$ is not a submodular function of $S$.

All omitted proofs and some additional details are  presented in the full version
of the paper \url{www.dropbox.com/sh/r6i10b6s18gduap/AAD-4ZUScaq8_-RBeDtCsp2oa?dl=0}.

\vspace{-0.1in}
\section{$\infprob{}$: Some Basic Results}
\vspace{-0.05in}
\subsection{Model and Problem Definition}
\vspace{-0.05in}
\label{sec:model}
Consider a graph $G$ with $n$ nodes and $m$ directed edges. This graph models a network of
influence, where an edge $(u,v)$ has a weight $0 \leq w(u,v) \leq 1$, indicating how likely $u$
influences $v$. The problem of interest is to analyze how influence is propagated in the network.
We use the well-studied {\em Independent Cascades (IC) model} with discrete time to model the spread of influence which we explain below
\cite{kkt-2003}.

At any given time $t$, the nodes have one of three states: \textit{active, newly active, inactive}.
At time $t$, let $A_t$ be the set of active nodes, $S_t$ be the set of newly active nodes, and the rest
be inactive. Each node $u \in S_t$ can activate each of its inactive neighbors $v$ with a
probability $w(u,v)$. Let $U_t$ be the set of  nodes activated in this step. At time
$t+1$, $A_{t+1} = A_t \cup U_t$, and $S_{t+1} = U_t$. Starting from an initial configuration
of a {\em seed set} $S = S_0$, we apply the process until time $\tau$ where $U_{\tau} = \varnothing$,
indicating no new nodes can be activated. We denote $I(S) = \abs{A_{\tau}}$ as the \textit{influence} of the set $S$.
More generally, we have a weight $wt(v)$ for each node, and $wt(I(S)) = \sum_{v\in A_{\tau}} wt(v)$ is the total weight
of the influence set $I(S)$. For simplicity, we will focus on the unweighted version of the problem; all our results hold for the weighted version as well, with natural changes in bounds. We note that $I(S)$ is a random variable that depends on $S$ (and the underlying
diffusion process).

% A fundamental problem that has been very well-studied is to  find the set $S$ of a given size
% $k$ such that the expected influence --- $\E[I(S)]$ --- is maximized
% (cf. Section \ref{sec:related}). This is referred to as the \infmax{} problem.
As mentioned in Section \ref{sec:intro}, the \infmax{} problem, maximizing $E[I(S)]$, is a coarse
optimization. In particular, it does not give  probabilistic guarantees on the influence
of the chosen seed set. To get a better idea of $I(S)$, we may need to
estimate the variance, or even to acquire the distribution
of $I(\cdot)$.  However, this task is usually quite difficult and costly.
% It is not clear, for example,
% how one can efficiently find a seed set that has ``high" expected influence and ``low" variance; such
% a set may not even exist depending on the requirements.

In this paper, we propose another measure which could be more useful: given
a threshold  probability $\delta$, find a seed set $S$ such that the $\delta$-quantile value
of $I(S)$ is maximized. This measure
% , unlike the expected influence measure,
gives direct probabilistic
guarantees on the random variable $I(S)$.\footnote{Note that one can obtain a one-sided probability bound
from expectation using Markov's inequality; but this usually quite weak.} More formally,
for some
set $S$, and a threshold $\delta$, define the following measure:

\begin{equation} \label{eq:M_delta}
  M_{\delta}(I(S)) = max\{a | \Pr[I(S) \geq a] \geq \delta \}
\end{equation}

%In essence, this is a pessimistic estimation, which is more robust to outliers.
\medskip
The new optimization problem, referred to as \infprob{} is defined in the following manner: given
an instance $(G=(V, E), k, \delta)$, find a (seed) set $S$ of size $k$ such that $M_{\delta}(I(S))$ is
maximized.
\begin{equation}
\begin{alignedat}{2}
  & \text{maximize} & \qquad & M_{\delta}(I(S)) \\
  & \text{subject to} & & \abs{S} = k.
\end{alignedat}
\end{equation}

The goal of this paper is to study the above optimization problem and give algorithms for it.

%In a general context, $M_{\delta}(\cdot)$ is a function of a random variable, i.e. $I(S)$.
%We will consider
%this general setting then apply for our influence problem. But first, we show some motivation
%for this pessimistic measurement in the context of influence problem.

\vspace{-0.05in}
\subsection{Comparison with \infmax{}}

As mentioned, the standard influence maximization problem, \infmax{}, is defined as
following \cite{v011a004}: given an instance $(G=(V, E), k)$, find a set $S\subseteq V$ of
size at most $k$ such that $E[I(S)]$ is maximized.

A natural question is whether one could find a solution to the problem of maximizing
$M_{\delta}(I(S))$, i.e., \infprob{}  by  solving \infmax{}. We show below that,
in general, the solutions of these two problems can be quite different.

\iffalse
\begin{lemma}
There exist instances $(G, k, \delta)$, for which $\frac{M_{\delta}(S^*)}{M_{\delta}(\hat{S})}$ is
arbitrarily large, where $S^*$ is an optimum solution to the $\infprob$ problem,
and $\hat{S}$ is an optimum solution to the $\infmax$ problem.
\end{lemma}
\fi

\begin{lemma}
\label{lem:example}
There exist instances $(G, k, \delta)$, for which $\frac{E[I(S^*)]}{M_{\delta}(I(S^*))}$ is
arbitrarily large, where $S^*$ is an optimum solution to the $\infmax$ problem.
\end{lemma}

\iffalse
\subsection{Sampling methods}
All works in the influence maximization literature require a basic task: finding the influence of
a node or a set of nodes by sampling the  diffusion process in the underlying graph. Generally,
there are two methods of sampling. The first method is to directly simulate the spread of influence
from a given seed set, based on the model under consideration. This is the most accurate method,
but  costly in practice. The second method is the \textit{triggering set} technique
\cite{v011a004}, which is more practical, and is applicable for
\textit{independent cascades} and \textit{linear threshold} models.
Below we describe the triggering set technique for sampling.

Let $G(V,E)$ be the underlying graph with the influence probability $w(u,v)$ for the edges $(u,v)
\in E$. A sample $G_i(V,E_i)$ is a random subgraph of $G$ where $E_i$ is a subset of $E$
constructed as follows: each edge $(u,v) \in E$ is chosen randomly with probability $w(u,v)$ to be
included in $E_i$.
\fi

\iffalse
let $\mathcal{G}$ be the set of all random graphs generated by activating all edges in $E$ with
their probabilities. An instance $g$ sampled from $\mathcal{G}$ is considered a sample for
influence of any nodes in $V$. This holds due to the independence of edge activation. A more
detailed discussion can be found in their original work.
\fi

\vspace{-0.05in}
\subsection{Hardness}

We observe that \infprob{} is NP-hard to approximate within a factor of $(1-1/e)$, which is similar to the hardness of the $\infmax$ problem.

\begin{lemma}
\label{lemma:hardness}
It is NP-hard to obtain an approximate solution to the \infprob{} problem, within a factor of $(1-1/e)$.
\end{lemma}

\vspace{-0.05in}
\subsection{Computing $M_{\delta}(I(S))$ for A Fixed Set $S$}
%: Monte-Carlo approximation}
\label{sec:sampling}

We show how to compute $M_{\delta}(I(S))$ for a {\em fixed} set $S$. This is a necessary ingredient
in computing the solution to the \infprob{} problem.
% which tries to find the $S$ (of given size $k$) that maximizes $M_{\delta}(I(S))$
It is known that computing $M_{\delta}(I(S))$ exactly even for a fixed set $S$ is \#P-Hard \cite{zhang:kdd14}.
However, we show that we can estimate $M_{\delta}(I(S))$ efficiently
by using Monte-Carlo sampling and binary search; this is crucial in designing our efficient
multi-criteria approximation algorithm of Section \ref{sec:algo}.
The sampling algorithm, \textsc{MDelta(I($\cdot$))}, and its proofs are in the full version of the paper;
here we just present the final bounds that we need later.

\iffalse
%\subsection{Hardness of computing $M_{\delta}(I(S))$}
\subsubsection{Monte-Carlo approximation of $M_{\delta}(I(S))$}
To estimate $M_{\delta}(I(S))$, it will be useful to consider the following general problem of estimating
a random variable that takes values between 0 and 1 (both included).
%Our goal is to estimate the {\em probability} that a random variable exceeds a certain value.
\begin{equation}
\begin{alignedat}{2}
  & 0 \leq X \leq 1 && \text{ a random variable} ,\\
  & 0 < \delta \leq 1 && \text{ a probability threshold} ,\\
  & \text{find} & \qquad & M_{\delta}(X) = sup\{a | \Pr[X \geq a] \geq \delta \}.
\end{alignedat}
\end{equation}

We  define $M_{\delta}(X)$  in a way that is valid for
both discrete and continuous distributions. Indeed, let $F_{X}$ be the cumulative distribution of
$X$, if $F_X$ is continuous, we can define: $M_{\delta}(X) = sup\{a | \Pr[X \geq a] = \delta \}$,
and the solution is given by: $M_{\delta}(X) = F^{-1}_X(1 - \delta)$.

In general, we cannot afford finding the distribution function, thus, we apply a Monte Carlo
sampling to give an $(\epsilon, \eta)$-approximation of $M_{\delta}(X)$.
Let $\widetilde{M}_{\delta}(X, \epsilon, \eta)$ be such an approximation, satisfying:
\begin{equation}
\label{eq:numr-mdelta}
  \widetilde M_{\delta}(X, \epsilon, \eta) \geq max \{ a | \Pr(X \geq a) \geq \delta
    - \eta \} - \epsilon
\end{equation}

The idea is to perform binary search to find the best value of $a$. Initially, we guess $a = 1/2$,
then verify if $\Pr(X \geq a) \geq \delta$. If the test is confirmed, we make the next guess as
$a = 3/4$, otherwise, we guess $a = 1/4$. Continue until the step of the guess smaller than
$\epsilon$.

The slack $\eta$ in the probability comes from the testing for $\Pr(X \geq a) \geq \delta$. For our
problem,  we will design a Monte-Carlo sampling algorithm. Consider one iteration, with some fixed $a$,  we have to decide: if $\Pr(X \geq a) \geq
\delta$, or $\Pr(X \geq a) < \delta$.
Take $N$ samples $X_1,X_2,\cdots,X_N$, where $N$ will be specified later. Let
$p = \Pr(X \geq a)$ be the unknown, fixed probability. Define $N$ indicator random variables $Z_i$,
where $Z_i = 1$ if the sample $X_i \geq a$, $0$ otherwise. Let $Z = \sum_{i=1}^{N}Z_i$. We have
$\mu_Z=\E[Z] = Np$. Using Chernoff bound\cite{Upfal-book}, we bound the empirical probability $\bar p
= \frac{Z}{N}$:
\begin{align*}
  & \Pr(|Z - Np| \geq \nu Np) \leq 2 exp \left( - \frac{\nu^2}{2+\nu} Np \right) \\
  \iff &\Pr \left( |\bar p - p| \geq \nu p \right) \leq
    2 exp \left( - \frac{\nu^2}{2+\nu} Np \right).
\end{align*}

With the desired  error $\eta = \nu p \iff \nu = \frac{\eta}{p}$, the tail bound becomes:
\begin{equation}
  \Pr \left( |\bar p - p| \geq \eta \right) \leq
    2 exp \left( - \frac{\eta^2}{2p + \eta} N \right)
  \leq 2 exp \left( - \frac{\eta^2 N}{3} \right).
\end{equation}

Given $|\bar p - p| \geq \eta$, clearly we have: $\bar p - \eta \geq \delta \implies
p \geq \delta$, and $\bar p + \eta < \delta \implies p < \delta$. This standard method leaves
an undecided region when $\bar p - \eta < \delta \leq \bar p + \eta$. To simplify the decision
rule, we allow some slack around $\delta$, and have the following decision rule:
\begin{align}
\begin{aligned} \label{eq:3}
  & \text{If} \enskip \bar p  \geq \delta && \implies \text{increase} \,\,\,\, a; \\
  & \text{If} \enskip \bar p < \delta && \implies \text{decrease} \,\,\,\, a.
\end{aligned}
\end{align}
%
\Cref{alg:1} shows the pseudo code for the above  algorithm. Next, we will show that it gives an (correct)
 $(\epsilon, \eta)$-approximation of  $M_{\delta}(X)$ (as defined in \ref{eq:numr-mdelta}). Then we will analyze the required number of samples to ensure correctness with high probability.
%
\begin{lemma}
\label{lem:ee-aprox}
\Cref{alg:1} finds an  $(\epsilon, \eta)$-approximation of  $M_{\delta}(X)$, assuming that all the Monte-Carlo tests are successful.
\end{lemma}
\begin{proof}
%With the assumption, we omit the discussion on number of samples.
It follows directly from the binary
search that the final value $a$ is within $\pm \epsilon$ of the maximum of $a$. The decision rules
in \cref{eq:3} is equivalent to: increase $a$ when $p \geq \delta - \eta$, decrease $a$ when
$p < \delta + \eta$. This holds for every iteration, thus, $\Pr(X \geq a)$ is guaranteed to be
at least $\delta - \eta$. $\Box$
\end{proof}
%
In our algorithm, $N$ is the number of samples in each iteration. Suppose we want to compute an
$(\epsilon, \eta)$-approximation of  $M_{\delta}(X)$ with  high (confidence) probability, we use the following lemma:
\begin{lemma}
\label{lem:num-samples}
\Cref{alg:1} finds an $(\epsilon, \eta)$-approximation of $M_{\delta}(X)$, with probability of success  at least
$(1 - \alpha \log(1/\epsilon))$ (for some given parameter $\alpha$), using
$N_{\mathit{total}} \geq \frac{3}{\eta^2}\ln{\frac{2}{\alpha}} \log{\frac{1}{\epsilon}}$ samples.
\end{lemma}
\begin{proof}
  For each iteration to be successful with probability at least $(1-\alpha)$, we bound the number of
samples (in one iteration) by Chernoff bound:
\begin{align} \label{eq:4}
  2 exp \left( - \frac{N \eta^2}{3} \right) \leq \alpha
    \iff N \geq \frac{3}{\eta^2}\ln{\frac{2}{\alpha}}.
\end{align}
The number of iterations is: $\log(1/\epsilon)$. Using union bound, we have the success probability for
the entire algorithm: $(1 - \alpha \log(1/\epsilon))$. Hence, given  parameters
$\delta, \epsilon, \eta, \alpha$,  \Cref{alg:1}   finds
$\widetilde{M}_\delta(X, \epsilon, \eta)$ with success probability at least $(1 - \alpha \log(1/\epsilon))$, and
requires at least $N_{\textit{total}} = \frac{3}{\eta^2}\ln{\frac{2}{\alpha}}
\log{\frac{1}{\epsilon}}$ samples. $\Box$
\end{proof}
%
One may ask why \cref{lem:num-samples} uses confidence of $(1 - \alpha \log(1/\epsilon))$, instead of the canonical form $(1 - \zeta)$ where $\zeta < 0$. Indeed, we choose the representation of $\alpha \log(1/\epsilon)$ to facilitate the analysis when applied to the graph influence problem, which is presented next.

\begin{algorithm}
  \caption{Approximate ${M_{\delta}(X)}$ with confidence
            $(1 - \alpha \log(1/\epsilon))$ }
  \label{alg:sampling}
  \begin{algorithmic}[1]
    \Function{MDelta}{$X[0,1]$, $\delta$, $\epsilon$, $\eta$, $\alpha$}
    \State $N \gets \frac{3}{\eta^2}\ln{\frac{2}{\alpha}}$
    \State $l \gets \epsilon$
    \State $h \gets 1$

    \While {$h - l \geq \epsilon$}
      \State $\mathit{mid} = (h + l) / 2$
      \State $X_1,\cdots,X_N \gets \textit{ samples of } X $
      \State $\bar p = \frac{1}{N}(\# X_i, X_i \geq \mathit{mid})$
      \If {$\bar p \geq \delta $} $l \gets \mathit{mid}$
      \Else \enskip $h \gets \mathit{mid}$
      \EndIf
    \EndWhile
    \Return $l$
    \EndFunction
  \end{algorithmic}
\end{algorithm}
%\subsection{Estimation of $M_{\delta}(I(S))$}
\begin{theorem}
\label{thr:numsmpls}
  On a graph of size $n$ nodes,  a given seed set $S$, for  given parameters
   $\delta$ and $\eta$ (assumed
  to be small constants)
  \Cref{alg:1} finds an $(\epsilon, \eta)$-approximation of $M_{\delta}(I(S))$ using $O((\log n)^2)$ samples with success probability
  $1- \Omega\left(\frac{1}{n}\right)$.
\end{theorem}
\begin{proof}
Consider the choice of parameters when applying \Cref{alg:1} to estimate $I(S)$. Because $1 \leq
I(S) \leq n$, we normalize $I(S)$ into the range $(0,1]$, and chose $\epsilon = 1/n$.
For the algorithm to succeed with probability at least $(1 - 1/n)$,
requires $\alpha = \frac{1}{n \log n}$ (by \cref{lem:num-samples}).
 Hence the number of
samples is:$\frac{3}{\eta^2}\log{n}\ln{\frac{n\log n}{2}} = O(\log^2 n)$.
$\Box$
\end{proof}
\fi
%
For some random variable $X$, define $\widetilde{M}_{\delta}(X, \epsilon, \eta)$ to be
an $(\epsilon, \eta)$-approximation of $M_{\delta}(X)$ if:
%\onlyShort{$\widetilde M_{\delta}(X, \epsilon, \eta) \geq max \{ a | \Pr(X \geq a) \geq \delta
 %   - \eta \} - \epsilon$.}
\begin{equation}
\label{eq:numr-mdelta}
  \widetilde M_{\delta}(X, \epsilon, \eta) \geq max \{ a | \Pr(X \geq a) \geq \delta
    - \eta \} - \epsilon
\end{equation}
%
\begin{theorem} [Sampling Bound for $M_{\delta}(I(S))$]
\label{thr:numsmpls}
  On a graph of size $n$ nodes,  a given seed set $S$, for  given parameters
   $\delta$ and $\eta$
  there is a Monte-Carlo randomized algorithm that finds an $(\epsilon, \eta)$-approximation of $M_{\delta}(I(S))$
  with probability of success  at least
$(1 - \alpha \log(1/\epsilon))$ (for some given parameter $\alpha$), using
$N_{\mathit{total}} \geq \frac{3}{\eta^2}\ln{\frac{2}{\alpha}} \log{\frac{1}{\epsilon}}$ samples.
  In particular, if  $\delta$ and $\eta$ are small constants, then only
  $O((\log n)^2)$ samples are needed to obtain success probability
  $1- O\left(\frac{1}{n}\right)$.
\end{theorem}
%
\subsection{A Simple Greedy Heuristic for \mdelta}
\label{sec:heu}

Motivated by the greedy algorithm of \cite{kkt-2003} for the $\infmax{}$ problem, we first consider a greedy algorithm for \mdelta{}: start with an empty set $S$, and proceed in
$k$ steps; in each step, select the node $v\not\in S$ which yields the highest increment in $\mdelta$, and add it to $S$. The pseudocode is shown in \Cref{alg:2}. The algorithm uses  $O(k(\log n)^2)$ samples overall, since it uses
$k$ applications of \textsc{MDelta(I($\cdot$))}  and each application uses $O(\log^2 n)$ samples to estimate the influence
of the current seed set with high probability (cf. Theorem  \ref{thr:numsmpls}).

\iffalse
Motivated by the greedy algorithm of \cite{kkt-2003} for the $\infmax{}$ problem, we first consider a greedy algorithm for \mdelta{}: start with an empty set $S$, and proceed in
$k$ steps; in each step, select the node $v\not\in S$ which yields the highest increment in $\mdelta$, and add it to $S$. The pseudocode is shown in \Cref{alg:2}. The algorithm uses  $O(k(\log n)^2)$ samples overall, since it uses
$k$ applications of algorithm \textsc{MDelta(I($\cdot$))} where each uses $O(\log^2 n)$ samples. The success probability of the algorithm is by union bound on $k$ steps
(see Theorem \ref{thr:numsmpls}).
\fi
%
\iffalse
\begin{lemma}
\label{thr:heu}
 \Cref{alg:2}
   uses $O(k(\log n)^2)$ samples overall assuming that
   is at least $1- \Omega\left(\frac{k}{n}\right)$.
\begin{proof}
  We proceed in a similar fashion to the proof of Theorem \ref{thr:numsmpls}. However, we need the union bound for
  high probability of success over the course of the algorithm, which includes $k$ iterations.
  Since $k$ is a bounded constant, we only have to ensure that the error probability bound for one iteration is small enough.
  Consider iteration $i$, where the incremental seed set size is $i$. The algorithm needs
  to estimate $M_{\delta}(\cdot)$ for $(n - i \approx n)$ candidates sets. For the union bound to hold, we require each estimation has a confidence of $O(1-1/n^2)$, then, by \cref{lem:num-samples}, $\alpha = \frac{1}{n^2 \log n}$.
  The total number of samples is thus:
  \begin{equation}
      k \frac{3}{\eta^2}\log{n}\ln{\frac{n^2\log n}{2}} = O(k\log^2 n). \Box
  \end{equation}
\end{proof}
\end{lemma}
\fi
%
\begin{algorithm}[t]
\footnotesize
\caption{\infprobheu{} - Heuristic Incremental $k$-influence-set}
\label{alg:2}
\begin{algorithmic}[1]
  \Function{ProbInf-Heu}{$G(V,E), k, \delta, \eta$}
  \State $\epsilon \gets \frac{1}{n}$
  \State $\alpha \gets \frac{1}{n\log n}$
  \State $S \gets \varnothing$
  \State $V' \gets V$ %\textit{ set of nodes}
  \While {$|S| < k$}
    \State $u \gets \argmax\limits_{v \in V'}$
      \Call{MDelta}{$I(S \cup \{v\}) - I(S), \delta, \eta, \epsilon, \alpha$}
    \State $S \gets S \cup \{u\}$
    \State $V' \gets V' - \{u\}$
  \EndWhile
  \Return $S$
  \EndFunction
\end{algorithmic}
\end{algorithm}
%\red{Gopal: Explain what guarantee on $\delta$ means? This should be explained in the first para when we explain
%the greedy algorithm.}
Algorithm~\ref{alg:2}  does not guarantee any approximation bound regarding the influence, since $M_{\delta}(I(\cdot))$
can be shown to be not a sub-modular set function\cite{zhang:kdd14} (a proof can be found in the full paper).
However, we find empirically that Algorithm \ref{alg:2} performs quite well on real graph data sets (Section \ref{sec:expt}).
%
%
% \vspace{-0.05in}
% \subsection{Non-submodularity of \mdelta}
% We note that \mdelta\ is non-submodular. The proof is by a constructive example, and can be found in \cite{zhang:kdd14}, with slightly different notations.
% We consider the weighted version of \infprob{}, with an additional constraint that the seed set $S\subseteq R$, where $R\subseteq V$ is fixed as part of the input.
% For a given seed set $S$, we define $\mdelta$ as the maximum value $a$ such that $\Pr[wt(I(S)) \geq a]\geq\delta$. The following example shows that $f(S) = \mdelta$ is not submodular.

% Consider a graph with vertices $V=\{a, b, c, d\}$, weights $wt(a) = wt(b) = wt(d) = 0$, $wt(c)=10$ and edges $E=\{(a, c),
% (b, c)\}$ (so $d$ is isolated). Consider $R=\{a, b, d\}$ as the set of potential seeds. Assume both edges in $E$ have probability $p$ and $\delta=p+\epsilon$, for a small $\epsilon$.

% We have $f(\{a\}) = 0$, since the edge $(a,c)$ is picked with probability $p$. Similarly,
% $f(\{b\}) = f(\{d\}) = 0$. We have $f(\{a, b\}) = 10$, because one of the edges survives with
% probability $2p-p^2 > p+\epsilon$. Therefore,
% \[
% 0 = f(\{a\}) - f(\phi) \leq f(\{a, b\}) - f(\{b\}) = 10,
% \]
% which implies $f(\cdot)$ is not submodular. So in this instance, a greedy algorithm, which adds a vertex $v$ that maximizes $f(S\cup\{v\}) - f(S)$ might end up
% picking the set $\{d, a\}$, which is arbitrarily worse than $\{a, b\}$.


%%%%%%
%bbb
\vspace{-0.1in}
\section{Related Work}
\label{sec:related}
The works by \cite{kkt-2003}, \cite{v011a004}, were the first to formulate the problem of influence
spreading as a discrete
optimization problem. They consider two main diffusion models: \textit{independent cascade} (which we adopt in this paper) and
\textit{linear threshold}. Works in these models
include two main optimization problems: maximizing the expected influence with constraints on the
seed set (usually size), and minimizing the seed set (according to some measurement) to achieve a target expected
influence. These problems are at least NP-hard, \cite{v011a004}. In \cite{chen2010scalable}, it
was shown that computing the expected influence is \#P-hard. These results set a theoretical limit
on what can be done to attack the problems.

The approach proposed by \cite{kkt-2003}, using submodular
set function and greedy heuristic, gives a $(1-1/e-\epsilon)$-approximation to the maximizing the expected
influence problem. The error $\epsilon$ is due to Monte Carlo estimation of the expected
influence, which is the main issue in practice, where large real world graphs discourage
excessive sampling. \cite{borgs2014maximizing} propose an algorithm with
nearly optimal theoretical runtime of $O((m+n)k\epsilon^{-2}\log n)$ while retaining the same
approximation guarantee. The technique is to sample \textit{reversed} influence, which is adopted
and improved upon in other works, e.g., \cite{tang2014influence}, \cite{tang2015influence},
\cite{nguyen2016stop}. While reversed influence sampling has a theoretical guarantee for the
expectation problem, it is not extendable to our probabilistic \mdelta.
Another approach, proposed by \cite{lucier:kdd15}, is to
estimate expected influence as a Riemann sum, using $O(n \epsilon^2 \text{polylog}(n))$ samples
which can be implemented in  parallel by using MapReduce.

Other models are also studied, for example, fixed threshold models \cite{chen2009approximability},
\cite{goldberg2013diffusion}, time-restricted diffusion model
\cite{goyal2013minimizing}, \cite{chen2012time}, \cite{dinh2014cost},
continuous-time diffusion model
\cite{du2013scalable}, and diffusion in dynamic network \cite{tong2017adaptive}.

The work closest to ours is presented by \cite{zhang:kdd14}. They
propose to measure influence with a probabilistic guarantee; their  goal is to find a minimum-sized seed set   that achieves a given
target influence with a specified probability. They show that their problem is \#P-hard, and give an
approximation algorithm. The algorithm uses expected influence as the criteria (which requires high number of samples) to select
members of the seed set, and then verifies the probability condition.
They show that the size of the output seed set, compared to the
optimal one, incurs both a multiplicative error of $(\ln n + O(1))$ and an additive error
of $O(\sqrt{n})$, under the assumption that the standard deviation of the influence is
$O(\sqrt{n})$. In contrast, our goal is different ---
we want to maximize influence with probabilistic guarantee, with a constraint on the seed set
size. Our multi-criteria approximation does not incurs additive error, and does not relies
on assumption about the distribution.

\section{Computing $M_{\delta}(I(S))$ for a fixed set $S$}
\label{sec:sampling}

In this section, we show how to compute $M_{\delta}(I(S))$ for a {\em fixed} set $S$. This is a necessary ingredient in computing the solution to the \infprob{} problem, which
tries to find the $S$ (of given size $k$) that maximizes $M_{\delta}(I(S))$.  It can be shown  that computing $M_{\delta}(I(S))$ exactly even for a fixed set $S$ is \#P-Hard \cite{zhang:kdd14}. However, we show  that we can estimate  $M_{\delta}(I(S))$ by using Monte-Carlo sampling.
%\subsection{Hardness of computing $M_{\delta}(I(S))$}
\subsection{Monte-Carlo approximation of $M_{\delta}(I(S))$}
To estimate $M_{\delta}(I(S))$, it will be useful to consider the following general problem of estimating
a random variable that takes values between 0 and 1 (both included). 
%Our goal is to estimate the {\em probability} that a random variable exceeds a certain value.
\begin{equation}
\begin{alignedat}{2}
  & 0 \leq X \leq 1 && \text{ a random variable} ,\\
  & 0 < \delta \leq 1 && \text{ a probability threshold} ,\\
  & \text{find} & \qquad & M_{\delta}(X) = sup\{a | \Pr[X \geq a] \geq \delta \}.
\end{alignedat}
\end{equation}

We  define $M_{\delta}(X)$  in a way that is valid for
both discrete and continuous distributions. Indeed, let $F_{X}$ be the cumulative distribution of
$X$, if $F_X$ is continuous, we can define: $M_{\delta}(X) = sup\{a | \Pr[X \geq a] = \delta \}$,
and the solution is given by: $M_{\delta}(X) = F^{-1}_X(1 - \delta)$.

In general, we cannot afford finding the distribution function, thus, we apply a Monte Carlo
sampling to give an $(\epsilon, \eta)$-approximation of $M_{\delta}(X)$.
Let $\widetilde{M}_{\delta}(X, \epsilon, \eta)$ be such an approximation, satisfying:
\begin{equation}
\label{eq:numr-mdelta}
  \widetilde M_{\delta}(X, \epsilon, \eta) \geq max \{ a | \Pr(X \geq a) \geq \delta
    - \eta \} - \epsilon
\end{equation}

The idea is to perform binary search to find the best value of $a$. Initially, we guess $a = 1/2$,
then verify if $\Pr(X \geq a) \geq \delta$. If the test is confirmed, we make the next guess as
$a = 3/4$, otherwise, we guess $a = 1/4$. Continue until the step of the guess smaller than
$\epsilon$.

The slack $\eta$ in the probability comes from the testing for $\Pr(X \geq a) \geq \delta$. For our
problem,  we will design a Monte-Carlo sampling algorithm. Consider one iteration, with some fixed $a$,  we have to decide: if $\Pr(X \geq a) \geq
\delta$, or $\Pr(X \geq a) < \delta$.
Take $N$ samples $X_1,X_2,\cdots,X_N$, where $N$ will be specified later. Let
$p = \Pr(X \geq a)$ be the unknown, fixed probability. Define $N$ indicator random variables $Z_i$,
where $Z_i = 1$ if the sample $X_i \geq a$, $0$ otherwise. Let $Z = \sum_{i=1}^{N}Z_i$. We have
$\mu_Z=\E[Z] = Np$. Using Chernoff bound\cite{Upfal-book}, we bound the empirical probability $\bar p
= \frac{Z}{N}$:
\begin{align*}
  & \Pr(|Z - Np| \geq \nu Np) \leq 2 exp \left( - \frac{\nu^2}{2+\nu} Np \right) \\
  \iff &\Pr \left( |\bar p - p| \geq \nu p \right) \leq
    2 exp \left( - \frac{\nu^2}{2+\nu} Np \right).
\end{align*}

With the desired  error $\eta = \nu p \iff \nu = \frac{\eta}{p}$, the tail bound becomes:
\begin{equation}
  \Pr \left( |\bar p - p| \geq \eta \right) \leq
    2 exp \left( - \frac{\eta^2}{2p + \eta} N \right)
  \leq 2 exp \left( - \frac{\eta^2 N}{3} \right).
\end{equation}

Given $|\bar p - p| \geq \eta$, clearly we have: $\bar p - \eta \geq \delta \implies
p \geq \delta$, and $\bar p + \eta < \delta \implies p < \delta$. This standard method leaves
an undecided region when $\bar p - \eta < \delta \leq \bar p + \eta$. To simplify the decision
rule, we allow some slack around $\delta$, and have the following decision rule:
\begin{align}
\begin{aligned} \label{eq:3}
  & \text{If} \enskip \bar p  \geq \delta && \implies \text{increase} \,\,\,\, a; \\
  & \text{If} \enskip \bar p < \delta && \implies \text{decrease} \,\,\,\, a.
\end{aligned}
\end{align}
%
\Cref{alg:1} shows the pseudo code for the above  algorithm. Next, we will show that it gives an (correct)
 $(\epsilon, \eta)$-approximation of  $M_{\delta}(X)$ (as defined in \ref{eq:numr-mdelta}). Then we will analyze the required number of samples to ensure correctness with high probability.
%
\begin{lemma}
\label{lem:ee-aprox}
\Cref{alg:1} finds an  $(\epsilon, \eta)$-approximation of  $M_{\delta}(X)$, assuming that all the Monte-Carlo tests are successful.
\end{lemma}
\begin{proof}
%With the assumption, we omit the discussion on number of samples.
It follows directly from the binary
search that the final value $a$ is within $\pm \epsilon$ of the maximum of $a$. The decision rules
in \cref{eq:3} is equivalent to: increase $a$ when $p \geq \delta - \eta$, decrease $a$ when
$p < \delta + \eta$. This holds for every iteration, thus, $\Pr(X \geq a)$ is guaranteed to be
at least $\delta - \eta$. $\Box$
\end{proof}
%
In our algorithm, $N$ is the number of samples in each iteration. Suppose we want to compute an
$(\epsilon, \eta)$-approximation of  $M_{\delta}(X)$ with  high (confidence) probability, we use the following lemma:
\begin{lemma}
\label{lem:num-samples}
\Cref{alg:1} finds an $(\epsilon, \eta)$-approximation of $M_{\delta}(X)$, with probability of success  at least
$(1 - \alpha \log(1/\epsilon))$ (for some given parameter $\alpha$), using 
$N_{\mathit{total}} \geq \frac{3}{\eta^2}\ln{\frac{2}{\alpha}} \log{\frac{1}{\epsilon}}$ samples.
\end{lemma}
\begin{proof}
  For each iteration to be successful with probability at least $(1-\alpha)$, we bound the number of
samples (in one iteration) by Chernoff bound:
\begin{align} \label{eq:4}
  2 exp \left( - \frac{N \eta^2}{3} \right) \leq \alpha
    \iff N \geq \frac{3}{\eta^2}\ln{\frac{2}{\alpha}}.
\end{align}
The number of iterations is: $\log(1/\epsilon)$. Using union bound, we have the success probability for
the entire algorithm: $(1 - \alpha \log(1/\epsilon))$. Hence, given  parameters
$\delta, \epsilon, \eta, \alpha$,  \Cref{alg:1}   finds
$\widetilde{M}_\delta(X, \epsilon, \eta)$ with success probability at least $(1 - \alpha \log(1/\epsilon))$, and
requires at least $N_{\textit{total}} = \frac{3}{\eta^2}\ln{\frac{2}{\alpha}}
\log{\frac{1}{\epsilon}}$ samples. $\Box$
\end{proof}
%
One may ask why \cref{lem:num-samples} uses confidence of $(1 - \alpha \log(1/\epsilon))$, instead of the canonical form $(1 - \zeta)$ where $\zeta < 0$. Indeed, we choose the representation of $\alpha \log(1/\epsilon)$ to facilitate the analysis when applied to the graph influence problem, which is presented next.

\begin{algorithm}
  \caption{Approximate ${M_{\delta}(X)}$ with confidence
            $(1 - \alpha \log(1/\epsilon))$ }
  \label{alg:1}
  \begin{algorithmic}[1]
    \Function{MDelta}{$X[0,1]$, $\delta$, $\epsilon$, $\eta$, $\alpha$}
    \State $N \gets \frac{3}{\eta^2}\ln{\frac{2}{\alpha}}$
    \State $l \gets \epsilon$
    \State $h \gets 1$

    \While {$h - l \geq \epsilon$}
      \State $\mathit{mid} = (h + l) / 2$
      \State $X_1,\cdots,X_N \gets \textit{ samples of } X $
      \State $\bar p = \frac{1}{N}(\# X_i, X_i \geq \mathit{mid})$
      \If {$\bar p \geq \delta $} $l \gets \mathit{mid}$
      \Else \enskip $h \gets \mathit{mid}$
      \EndIf
    \EndWhile
    \Return $l$
    \EndFunction
  \end{algorithmic}
\end{algorithm}
%\subsection{Estimation of $M_{\delta}(I(S))$}
\begin{theorem} 
\label{thr:numsmpls}
  On a graph of size $n$ nodes,  a given seed set $S$, for  given parameters
   $\delta$ and $\eta$ (assumed
  to be small constants)
  \Cref{alg:1} finds an $(\epsilon, \eta)$-approximation of $M_{\delta}(I(S))$ using $O((\log n)^2)$ samples with success probability
  $1- \Omega\left(\frac{1}{n}\right)$.
\end{theorem}
\begin{proof}
Consider the choice of parameters when applying \Cref{alg:1} to estimate $I(S)$. Because $1 \leq
I(S) \leq n$, we normalize $I(S)$ into the range $(0,1]$, and chose $\epsilon = 1/n$.
For the algorithm to succeed with probability at least $(1 - 1/n)$,
requires $\alpha = \frac{1}{n \log n}$ (by \cref{lem:num-samples}). 
 Hence the number of
samples is:$\frac{3}{\eta^2}\log{n}\ln{\frac{n\log n}{2}} = O(\log^2 n)$.
$\Box$
\end{proof}
%\section{An Algorithm for \infprob{}}

\vspace{-0.1in}
\section{A Multi-criteria Approximation Algorithm for Computing \infprob{}}
\label{sec:algo}
\vspace{-0.05in}
%\red{Nguyen: I change alpha in bicrit to gamma, to avoid conflict with alpha in sampling. Should bicriteria be  tricriteria? we have: delta, influence, size. Relaxation on delta will be a fix factor of 1/2. So, for influence and size, we can discuss as gamma-beta bicriteria...}

Motivated by the hardness from Lemma \ref{lemma:hardness}, and the non-submodularity of $\mdelta$, we consider a multi-criteria approximation algorithm, which relaxes the seed set size $k$, as well as the probability parameter $\delta$.
%Observe that in algorithm 2, we perform a full binary search to find $M_{\delta}$ whenever we want to add the next element into $S$, and due to non-submodularity, we cannot reason about the quality of the resulted $S$. In this section, we try to analyse the quality of the approximation, by another approach. There are two main ideas: convert the problem into a series of bicriteria approximation, and introduce a submodular measurement.
Specifically, we consider the following variant of the $\mdelta$ problem: find $S$ such that $\mdelta \geq \gamma a $ and $|S| \leq \beta k$, where $a$ corresponds to the influence size for the optimal solution, and $\gamma < 1, \beta > 1$ are the relaxation parameters. The quantity $a$ will be determined through the binary search.
%(as done in the Monte-Carlo sampling algorithm \ref{alg:1}.

%To efficiently find a feasible $S$, we need to define a submodular measurement, which later converts back into $M_{\delta}$. We will construct an algorithm that finds an approximation solution $\hat S$ satisfying the following bicriteria:
%\begin{equation}
%\begin{aligned}
%  M_{\delta/2}(I(\hat S)) \geq \frac{1}{2} M_{\delta}(I(S^*)) = \frac{1}{2}c^* \\
 % \text{and} \qquad |S| \leq O(\ln c^*) k
%\end{aligned}
%\end{equation}

We use ideas from the result of Krause et al.~\cite{krause:jmlr08} on minimum cost submodular cover problem; our problem has crucial differences, because of which we cannot directly use their algorithm. Assume we have $N$ samples $G_1,\ldots, G_N$, where $G_i=(V, E_i)$ is the $i$th sample, obtained by the \textit{triggering set} technique (we discuss bounds on $N$ later in Theorem \ref{theorem:mulcritguarantee}). Let $F_i(S)$ denote the total influence due to
$S$ in sample $i$ -- this equals the sum of the component sizes containing $S$. Define
$
F_{i, \lambda}(S) = \min\{F_i(S), \lambda\}.
$

The probabilistic guarantee in $\mdelta$ requires finding $S$ that ensures that $|\{i: F_{i, \lambda}(S)\geq \lambda\}|\geq\delta N$. However, it is quite challenging to find a set $S$ with such a property in an incremental manner: a node $v$ might increase $F_{i, \lambda}(S)$ only in some samples $i$. Thus, it is not clear how to pick nodes each time that ensure influence in many samples is high, eventually. Instead, we will consider the average defined as
\[
F_{\lambda}(S) = \frac{1}{N}\sum_i F_{i, \lambda}(S).
\]
A critical observation is that $F_{i, \lambda}(\cdot)$ and thus $F_{\lambda}(\cdot)$ are submodular \cite{fujito2000approximation}.
Hence, greedy approach to maximize $F_{\lambda}(\cdot)$ will give a guaranteed approximation.
We show that maximizing $F_{\lambda}(S)$ is good enough due to the following property.
%We follow the general approach of Krause et al.~\cite{krause:jmlr08}, who study the following problem: given submodular functions $F_1,\ldots,F_N$, and a parameter $k$, find $S\subseteq V$ such that
%\[
%\max_{c, S, |S|\leq k} F_i(S)\geq c, \mbox{ for all $i$}
%\]
% Krause et al. reduce this to the minimum cost submodular cover problem, which involves finding a subset $S\subseteq V$ such that $F(S)=F(V)$ and $\mbox{cost}(S)$ is minimized, where $F$ is a submodular function. Our problem leads to a variant of the submodular cover problem, where the goal is to find $S$ of the minimum cost, such that $F(S)\geq C$, where $C$ is a parameter, which might be smaller than $F(V)$.

\begin{lemma}
\label{lemma:Flambda}
If $|\{i\mbox{ such that }F_{i,\lambda}(S)\geq \lambda\}|\geq \delta N$, then $F_{\lambda}(S)
\geq \delta\lambda$. On the other hand, if $F_{\lambda}(S)\geq \delta\lambda$, then
$|\{i\mbox{ such that }F_{i,\lambda}(S)\geq \delta\lambda/2\}|\geq \delta N/2$.
\end{lemma}

%\begin{lemma}
%If  $|\{i\mbox{ s.t. } F_{i,\lambda}(S)\geq \lambda'\}|\geq \delta N$, then $F_{\lambda}(S) \geq \delta\lambda'$.
%\end{lemma}

Lemma \ref{lemma:Flambda} motivates the following strategy to solve the $\mdelta$ problem, which is a variant of the minimum cost submodular cover problem \cite{fujishige}: find $S$ such that $F_{\lambda}(S)\geq \delta\lambda$ and $\text{cost}(S)=\sum_{j\in S} w_j$ is minimized. Note that the standard submodular cover problem requires $F(S)=F(V)$, where $V$ is the ground set.

%%\red{Gopal: change bicriteria to multi-criteria throughout, including the pseudocode.}

\smallskip

\noindent\textbf{Fast implementation of algorithm \multicritalgo{}.} The pseudocode is shown in Algorithm \ref{alg:3}. It uses binary search as follows. With a given $\lambda$, an iteration decides if it is feasible to construct $S$ such that $F_{\lambda}(S)=C=\delta \lambda$, with a $\log$ factor constraint on size of $S$. If it is the case, $\lambda$ is a feasible value, and $S$ is a feasible solution for $max F_{\lambda}(\cdot)$ problem. Starting with $\lambda=n/2$, the binary search increases or decreases $\lambda$ as per the feasibility. The output is a feasible value and the respective set. Theorem~\ref{theorem:submodcover} shows the approximation guarantee of this algorithm.

\begin{algorithm}[t]
\footnotesize
\caption{$k$-Influence set by \multicritalgo{}}
\label{alg:3}
\begin{algorithmic}[1]
  \Function{\multicritalgo{}}{$G(V,E),k,\delta,n, N$}
  \State $l \gets 1$
  \State $h \gets n$
  \For{$step \gets [1 \dots \log n]$}
    \State $S \gets \emptyset$
    \State $\lambda \gets (l+h)/2$
    \State $C \gets \delta\lambda$, and $F(S) \gets F_{\lambda}(S)$
      \While {$F(S) < C$}
          \State Pick node $j$ that minimizes $\frac{w_j}{F(S\cup\{j\}) - F(S)}$
        \State $S \gets S \cup \{j\}$
      \EndWhile
    \State feasible if $|S| \le k\ln(N \lambda)$ : $l \gets \lambda$
    \State infeasible if $|S| > k\ln(N\lambda)$: $h \gets \lambda$
  \EndFor
  \Return $S$
  \EndFunction
\end{algorithmic}
\end{algorithm}

Let $S^*, \lambda^*=\mbox{argmax}_{S, \lambda} |\{i\mbox{ such that }F_{i,\lambda}(S)\geq \lambda\}|\geq \delta N$, where the maximization is over sets $S$ of size at most $k$.

\begin{theorem}
\label{theorem:submodcover}
Let $S^*$ be an optimum solution, as defined above.
The set $S$ computed by  Algorithm~\ref{alg:3} satisfies: (1) $F_{\lambda}(S)\geq \lambda^*\delta$ and (2) $|S| = O(\ln{N\lambda})k$, where $k=|S^*|$.
\end{theorem}

Our proof of Theorem \ref{theorem:submodcover} is a variation of the proof by Wolsey
\cite{wolsey:combinatorica82}, but has a crucial difference from minimum submodular cover, in that $F(S)$ need not equal to $F(V)$. 
%We will follow the notation and structure of the proof by
%Mestre\cite{mestre}.
Let $\rho_j(S) = F(S\cup\{j\}) - F(S)$. First observe that the following IP
is feasible:
$\mbox{(IP) } \min  \sum_j w_j x_j \mbox{ such that}$ for all $S\subseteq V$, with
$x_j \in \{0, 1\}$ we have
$\sum_{j\not\in S} \rho_j(S) x_j \geq C - F(S)$.


% \begin{eqnarray*}
% \mbox{(IP) } \min && \sum_j w_j x_j \mbox{ such that}\\
% \sum_{j\not\in S} \rho_j(S) x_j &\geq& C - F(S), \mbox{ for all $S\subseteq V$}\\
% x_j &\in& \{0, 1\}
% \end{eqnarray*}

\begin{lemma}
\label{lem:ipfeasible}
The program (IP) is valid, i.e., the optimum solution $S_{IP}$ satisfies $F(S_{IP}) \geq C$
and $S_{IP}$ has the minimum cost.
\end{lemma}
%
The dual program of the linear relaxation of (IP) is the following
%
\begin{eqnarray*}
\mbox{(D) } \max && \sum_S (C-F(S)) y_S \quad \mbox{such that}\\
\sum_{S: j\not\in S} \rho_j(S) y_S &\leq& w_j \mbox{ for all $j$}\\
y_S &\geq& 0
\end{eqnarray*}

Observe that the algorithm actually picks element $j$ which minimizes $\frac{w_j}{\rho_j(S)}$.
We construct an approximate dual solution. Suppose the greedy algorithm picks
elements $\{j_1,\ldots, j_{\ell}\}$. Let $S_i = \{j_1,\ldots, j_i\}$. Define
\[
\theta_i =
\begin{cases}
\frac{w_{j_i}}{F(S_i) - F(S_{i-1})} = \frac{w_{j_i}}{\rho_{j_i}(S_{i-1})}, \mbox{ for $i<\ell$,}\\
\frac{w_{j_{\ell}}}{C - F(S_{i-1})}, \mbox{ for $i=\ell$.}
\end{cases}
\]

Define
\[
y_S =
\begin{cases}
\theta_1, \mbox{ if $S=S_0=\emptyset$,}\\
\theta_{i+1} - \theta_i, \mbox{ if $S=S_i$ for $0<i<\ell$,}\\
0, \mbox{ otherwise.}
\end{cases}
\]

Observe that the dual solution $y_{S_i}$ covers the cost of the solution $S_{\ell}$:
\begin{eqnarray*}
&&\sum_S (C-F(S))y_S = \sum_{i=0}^{\ell-1} (C-F(S_i))y_{S_i}\\
&=& \sum_{i=0}^{\ell-1} (C-F(S_i))(\theta_{i+1}-\theta_i)\\
&=& \sum_{i=1}^{\ell-1}\theta_i(F(S_{i}) - F(S_{i-1})) + \theta_{\ell}(C-F(S_{\ell-1})) \\
&=& \sum_{i=1}^{\ell-1} w_{j_i} + w_{j_{\ell}} = \text{cost}(S_{\ell}).
\end{eqnarray*}

Next, we observe that the dual constraints are approximately feasible. First, consider any
$j\in V - S_{\ell}$. We have

\begin{eqnarray*}
\sum_{S: j\not\in S} \rho_j(S)y_S &=& \sum_{i=0}^{\ell} \rho_j(S_i) y_{S_i} \\
&=& \rho_j(S_0)\theta_1 + \sum_{i=1}^{\ell-1} \rho_j(S_i)(\theta_{i+1}-\theta_i)\\
&=& \sum_{i=1}^{\ell-1} \theta_i(\rho_j(S_{i-1})) + \theta_{\ell}\rho_j(S_{\ell-1}) \\
&\leq& \sum_{i=1}^{\ell-1} w_j \frac{\rho_j(S_{i-1}) - \rho_j(S_i)}{\rho_j(S_{i-1})} + w_j,
\end{eqnarray*}
because $\theta_i\leq \frac{w_j}{\rho_j(S_{i-1})}$ by construction, for each $i\leq\ell$.

For $x\in(0, \rho_j(\emptyset))$, define
$h(x) = \frac{1}{\rho_j(S_{i-1})}$, if $\rho_j(S_i) < x \leq \rho_j(S_{i-1})$. Let
\[
\delta = \min_{S: \rho_j(S)>0} \rho_j(S) = \min_{S:\rho_j(S)>0} F(S+j) - F(S) \geq \frac{1}{N}.
\]
%
Therefore,
\begin{eqnarray*}
&&\sum_{i=1}^{\ell-1} \frac{\rho_j(S_{i-1}) - \rho_j(S_i)}{\rho_j(S_{i-1})} = \int_0^{\rho_j(\emptyset)} h(x) dx \\
&\leq& \int_0^{\delta}\frac{1}{\delta} dx + \int_{\delta}^{\rho_j(\emptyset)} \frac{1}{x} dx
= 1 + \ln \frac{\rho_j(\emptyset)}{\delta}
\leq 1 + \ln (N\lambda),
\end{eqnarray*}
since $\rho_j(\emptyset)\leq F(j) \leq \lambda$.
This implies
\[
\sum_{S: j\not\in S}\rho_j(S) y_S \leq (2+\ln{N\lambda}) w_j.
\]

Next, suppose $j\in S_r$ for some $r$. Then, $\rho_j(S_{r})= F(S_{r+1}) - F(S_r)=0$. Let $r'<r$ be
the largest index such that $\rho_j(S_{r'})>0$. In that case, the dual constraint for $j$ can be
written as
\begin{eqnarray*}
\sum_{S: j\not\in S} \rho_j(S)y_S = \sum_{i=0}^{r'} \rho_j(S_i) y_{S_i}
= \sum_{i=1}^{r'+1} \theta_i(\rho_j(S_{i-1}) - \rho_j(S_i))\\
\leq \sum_{i=1}^{r'+1} w_j \frac{\rho_j(S_{i-1}) - \rho_j(S_i)}{\rho_j(S_{i-1})}
\leq 1+\ln{N\lambda},
\end{eqnarray*}
as before. Therefore, $\frac{1}{\alpha}y$ is a feasible solution, for $\alpha = 2+\ln{N\lambda}$, which  completes the proof for Theorem
\ref{theorem:submodcover}.
$\Box$

Finally, we combine this with the right number of samples (applying Theorem \ref{thr:numsmpls} using $\Theta(n)$ samples), and put everything together to obtain the following theorem.

\begin{theorem}
\label{theorem:mulcritguarantee}
Let $S^*$ denote the optimum solution to the \infprob{} problem.
Let $N=\Theta(\frac{c}{\epsilon^2\delta}n)$ samples, for a constant $c$. Then,  \multicritalgo{} gives a solution $S$ such that $M_{\delta/2}(I(S)) \geq \frac{(1-\epsilon)\delta}{2} M_{\delta}(I(S^*)$.
\end{theorem}
The factor $1/2$ for $\delta$ in the approximation solution comes from Lemma \ref{lemma:Flambda}.
Note that we need $\Theta(n)$ samples so that all subsets (there can be exponential number of them)
are accurately sampled with high probability by a union bound (applying Theorem \ref{thr:numsmpls}).
However,  our experimental results show that, in practice, we need substantially less number of samples  ---
$\Theta(\log n)$ --
to guarantee a good approximation.

% \begin{proof}
% Consider any set $S$, and let $\lambda=\mdelta$. Then, $\Pr[F_{i, \lambda}(S)\geq \lambda] =\delta$. By standard concentration using Chernoff bounds, $\Pr[F_{\lambda}(S)<(1-\epsilon)\lambda]\leq e^{-2n}$. Therefore, for all sets $S$, we have $F_{\lambda}(S)\geq(1-\epsilon)\lambda$ with probability at least $1-d^{-n}$, for a constant $d$.

% Let $\lambda^*$
% \end{proof}

% \noindent
% \textbf{Number of samples.}
% Let $S^*$ denote the optimal solution, let $c^* = M_{\delta}(I(S^*))$ denote the optimal value. Let
% \[
% F^N_{\lambda}(S) = \frac{1}{N}\sum_i F_{i, \lambda}(S).
% \]

% What we need to show
% \begin{itemize}
%     \item
% If $N=\Omega(\frac{c}{\epsilon^2\delta}n)$, then $F_{\lambda}^N(S)$ is within a $(1\pm\epsilon)$ factor of $\mdelta$, for each set $S$
% \item
% Let $S$ be the solution returned by \multicritalgo{}. Then, $F_{\lambda}^N(S) \geq \delta F_{\lambda}^N(S^*)/2$
% \item
% With high probability, we have $F_{\lambda}^N(S)\in(1\pm\epsilon)\mdelta)$ and $F_{\lambda}^N(S^*)\in (1\pm\epsilon)c^*$.
% \item
% Therefore, $\mdelta \in (1\pm 2\epsilon)\delta c^*/2$
% \end{itemize}

% To avoid cluttering details, let's first assume that the number of samples $N$ is large enough such that $M_{\delta}(I(\cdot))$ can be estimated with precise $\delta$. Condition on that, we have the following corollaries.
% \begin{corollary}
%   If $F_{\lambda}(S) \geq \delta \lambda$ then:
%   $M_{\delta/2}(I(S)) \geq \frac{1}{2}\delta \lambda$.
% \begin{proof}
%   Apply the ``only-if'' part of \red{Lemma 5.2}. $\blacksquare$
% \end{proof}
% \end{corollary}

% \begin{corollary}
%   In Algorithm \ref{alg:3}, if $\lambda < c^*$, then $\lambda$ is feasible.
% \begin{proof}
%   Since $F^N_{\lambda}(S^*) > \delta \lambda$, the while loop, with relax on $|S|$ by \red{Theo 5.2}, will successfully construct $S$ with $F^N_{\lambda}(S)=\delta \lambda$. $\blacksquare$
% \end{proof}
% \end{corollary}

% \begin{corollary}
%   If Algorithm \ref{alg:3} returns a feasible value $\lambda^{fsb}$ then $\lambda^{fsb} \geq c^*/2}$.
% \begin{proof}
%   If $\lambda^{fsb} > n/2$: Trivially holds.
%   If $\lambda^{fsb} < n/2$: By the binary search, $2\lambda^{fsb}$ must be infeasible. Which, by contraposition of \red{Cor 5.2}, implies $c^* < 2 \lambda^{fsb}$. $\blacksquare$.
% \end{proof}
% \end{corollary}

% \begin{corollary}
%   Algorithm \ref{alg:3} gives a set S such that:
%   \[
% F^N_{\lambda}(S) \geq \delta c^* /2
% \]
% \begin{proof}
%   Since $I(S)$ is in the range $[1,n]$, Algorithm \ref{alg:3} will always return some feasible value $\lambda$ and the corresponding set $S$. We have: $F^N_{\lambda}(S^*) \geq \delta \lambda$. By \red{Cor 5.3}, the proof is completed. $\blacksquare$
% \end{proof}
% \end{corollary}

% \begin{theorem}
%   Using $O(\eta^{-2} \log^2 n )$ samples, with high probability of success, \bicritalgo{} outputs a seed set
%   $S$, with $|S| = O(\log^2 n)|S^*|$ and the following guarantee approximation:
%   \[
%   M_{\frac{\delta - 3\eta}{2}}(I(S)) \geq \frac{1}{4}(\delta - \eta)c^*
%   \]
% \begin{proof}
%   Assuming that $M_\delta(I(\cdot))$ is estimated with a slack of $\eta$ around $\delta$. Since we design our sampling with absolute error $\eta$, this slack also holds for any value of $\delta$.

%   Adjust \red{Cor 5.4} with $\eta$:
%   $F^N_{\lambda}(S) \geq (\delta - \eta) c^* /2$. Then apply \red{Cor 5.1} with adjustment, we have:
%   \[
%   M_{\frac{\delta - 3\eta}{2}}(I(S)) \geq \frac{1}{4}(\delta - \eta)c^*
%   \]
%   \bicritalgo{} takes $\log n$ iteration. In each iteration, it incrementally constructs a set of size $O(\ln(N\lambda)k$. Assume that $N = o(n)$, and since $\lambda < n$, the set size is: $O(ln(n))$.
%   Using \red{lemma 4.2}, let $\alpha=(n^2 \ln n)^{-1}$. The set is constructed incrementally, by the argument similar to the proof of \red{Theo 5.1}, union bound of failure for one iteration is:
%   $\alpha n \ln(n)$ and union bound of failure for $\log n$ iteration is $\frac{1}{n}$.
%   Using \red{lemma 4.2}, plugin $\alpha$ to have $N=O(\eta^{-2}\log n \ln(n^2 \ln n)) = O(\eta^{-2}\log^2 n)$. This satisfies the assumption that $N=o(n)$. $\blacksquare$
% \end{proof}
% \end{theorem}

% We observe that the value of the dual solution is equal to the cost of the primal solution computed using greedy:
% \begin{eqnarray*}
% \sum_S (C-F(S))y_S &=& \sum_{i=0}^T (C-F(S_i)) y_{S_i}\\
% \end{eqnarray*}
% \section{Max-min version of Influence Maximization}
% This is a max-min version of the Influence maximization problem, which is an expectation version. Both are used in the stochastic optimization literature. The max-min version corresponds to $\delta=1$ of the problem, and is also interesting.
% Let $G_i=(V, E_i)$, $i=1,\ldots, N=\Theta(n^2)$, be random samples from a graph $G=(V, E)$,
% where each edge $e\in E$ is sampled with probability $p(e)$.
% Let $x(v)$ be an indicator variable which is $1$ if vertex $v$ is picked in the seed set.
% Let $\mathcal{C}(i)$ be the set of connected components in $G_i$. For component
% $C\in\mathcal{C}(i)$, let $y(C)$ be an indicator variable which is $1$ if
% some vertex from $C$ is in the seed set, i.e., if $C\cap S\neq\emptyset$.
% Let $I(S, i)=\cup_{C\in \mathcal{C}(i)} |C|y(C)$ be the influence set size in the sample $G_i$.
% Let $I(S) = \max_i I(S, i)$ be the maximum influence over all samples.
% We have the following linear program.
% \begin{eqnarray*}
% \max && \lambda \mbox{ such that}\\
% \sum_{C\in \mathcal{C}(i)} |C|y(C)  &\geq& \lambda, \mbox{ for all $i$}\\
% y(C) &\leq& \sum_{v\in C} x(v),\ \forall\mbox{$i$, $C\in \mathcal{C}(i)$}\\
% \sum_v x(v) &\leq& k\\
% x, y &\in& [0, 1]^n
% \end{eqnarray*}
% \begin{eqnarray*}
% \sum z(i) &\geq& \delta N\\
% \sum_{C\in \mathcal{C}(i)} |C|y(C)  &\geq& \lambda z(i), \mbox{ for all $i$}\\
% y(C) &\leq& \sum_{v\in C} x(v),\ \forall\mbox{$i$, $C\in \mathcal{C}(i)$}\\
% \sum_v x(v) &\leq& k\\
% x, y &\in& [0, 1]^n
% \end{eqnarray*}

% Rounding: choose $k$ nodes, each with probability $x(v)/k$\\
% Prob comp $C$ in sample $i$ is covered $\geq (1-1/e)y(C)$

% We select a set $S$ of size $\ell$, by picking a random vertex $v\in V$ with probability $x(v)/k$
% each time, with replacement. Let $OPT(k)$ denote the influence resulting from the
% optimum solution using a seed set of size $k$.

% \begin{lemma}
% Let $S$ be the seed set chosen by randomized rounding. Then, choosing
% $|S|=\ell=\frac{ckn\log{n}}{\epsilon^2OPT(k)}$ ensures that $I(S)\geq (1-\epsilon)(1-1/e)OPT(k)$.
% \end{lemma}
% \begin{proof} (Sketch)
% We split $S$ into $\ell/k$ sets $S_1,S_2,\ldots,S_{\ell/k}$ of size $k$ each.
% Consider a component $C\in\mathcal{C}(i)$.
% A vertex $v$, chosen randomly with probability $x(v)/k$, is in $C$ with probability
% $\sum_{v\in C} x(v)/k\geq y(C)/k$. Therefore, the probability that a randomly chosen
% vertex does not hit $C$ is at most $1-y(C)/k$. This implies the probability that
% no vertex in $S_j$ hits $C$ is at most $(1-y(C)/k)^k$. Therefore,
% \[
% \Pr[C\cap S_j\neq\emptyset] \geq 1 - (1-y(C)/k)^k \geq (1-1/e)y(C)
% \]

% This implies
% \[
% E[I(S_j, i)] \geq (1-1/e)\sum_{C\in\mathcal{C}(i)} |C|y(C) \geq (1-1/e)\lambda \geq (1-1/e)OPT(k)
% \]

% By Hoeffding's, we have
% \[
% \Pr[I(S_j, i) \leq (1-\epsilon)(1-1/e)OPT(k)] \leq 2exp(-\frac{\epsilon^2(1-1/e)OPT(k)}{2n})
% \]

% Therefore, we have
% \[
% \Pr[\mbox{for all $j\leq\ell/k$, }I(S_j, i) \leq (1-\epsilon)(1-1/e)OPT(k)] \leq 2exp(-\frac{\epsilon^2(1-1/e)OPT(k)\ell}{2n})
% \]
% This implies if $\ell=\Omega(\frac{n\log{n}}{\epsilon^2 OPT(k)})$, with probability
% at least $1-1/n^3$, we have
% $I(S=\cup_j S_j, i) \geq (1-\epsilon)(1-1/e)OPT(k)$. This implies
% $I(S) \geq (1-\epsilon)(1-1/e)OPT(k)$, with probability at least $1-1/n$.
% \end{proof}

% \noindent
% \textbf{Remark.} The above statement is useful only if $OPT(k)$ is large, i.e., $OPT(k)=\Omega(n)$, in which case it suffices to select $O(k\log{n})$ seeds. In general, maybe we will need to select sub-linear number of seeds.



% \subsection{Using Lucier's algorithm}

% We follow the notation from Lucier et al. \cite{lucier:kdd15}, with small modifications. For any vertex $i$, define
% \[
% \pi(i) = \sum_{t\in\{1, (1+\epsilon),\ldots\}} \frac{\epsilon}{1+\epsilon} t\Pr[I(i)> t]
% \]

% Define $\pi_t(i)=\Pr[I(i)>t]$ and $\pi_t(i, L)$ as the fraction of $L$ samples, in which the influence from $i$ is at least $t$.

% \begin{algorithm}
% \caption{InfEst}
% \label{alg:infest}
% \begin{algorithmic}[1]
%   \Function{InfEst}{$G(V,E)$}
%   \State $S \gets V$
%   \For{$\tau\in\{n, n/(1+\epsilon),\ldots,1\}$}
%     \For{$t\in\{\tau, (1+\epsilon)\tau,\ldots,n\}$}
%       \State Generate $L_t=\frac{8t\log^3{n}}{\tau\epsilon^2}$ random edge subgraphs
%       \State Each node $i\in S$ computes $\pi_t(i, L_t)$ in parallel
%     \EndFor
%     \State Each node $i\in S$ computes $x_i = \sum_t \frac{\epsilon}{1+\epsilon}t\pi_t(i, L)$ in parallel
%     \If{$i\in S$ and $x_i\geq (1-2\epsilon)\tau$}
%     \State $S=S-\{i\}$
%       \State $I(i)=\tau$
%   \EndIf
%   \EndFor

%   \State return $\max_i I(i)$
%   \EndFunction
% \end{algorithmic}
% \end{algorithm}

% The following statements follow from \cite{lucier:kdd15}
% \begin{itemize}
% \item
% The solution returned by InfEst$(G)$ is within $O(1)$ factor of the optimum
% \item
% Number of times connected components is run?
% \end{itemize}

\vspace{-0.1in}
\section{Empirical Evaluation}
\label{sec:expt}

\begin{figure*}[t]
% \begin{subfigure}{.40\textwidth}
%   \subimport*{graphs/}{fb_a001_05.tex}
%   \caption{Facebook data set.}
% \end{subfigure}
% \hfill
% \begin{subfigure}{.40\textwidth}
%   \subimport*{graphs/}{tw_a001_05.tex}
%   \caption{Twitter data set.}
% \end{subfigure}
% \\
\resizebox{\textwidth}{!}
{
\begin{subfigure}[t]{.45\textwidth}
  \captionsetup{skip=-8pt}
  \subimport*{graphs/}{fb_a001_05.tex}
  \caption{Facebook data set.}
  \label{fig:combine-a}
\end{subfigure}
\begin{subfigure}[t]{.45\textwidth}
  \captionsetup{skip=-8pt}
  \subimport*{graphs/}{tw_a001_05.tex}
  \caption{Twitter data set.}
  \label{fig:combine-b}
\end{subfigure}
%
\begin{subfigure}[t]{.45\textwidth}
  \captionsetup{skip=4pt}
  \subimport*{graphs/}{fb.tex}
  \caption{Facebook dataset.}
  \label{fig:combine-c}
\end{subfigure}
\begin{subfigure}[t]{.45\textwidth}
  \captionsetup{skip=4pt}
  \subimport*{graphs/}{tw.tex}
  \caption{Twitter dataset.}
  \label{fig:combine-d}
\end{subfigure}
}
\captionsetup{belowskip=-0.1in}
\caption{
\small
Comparison of three algorithms outputs. For $\infprobheu$ and $\multicritalgo$ algorithms, the seed sets are optimized with
$\delta=0.7$. Edges activation are random values in the range $[0.001,0.05]$.
Figures (a) and (b) show the average influence versus seed set size.
Figures (c) and (d) show the Empirical cumulative distribution function (ECDF) of influence of
seed sets of size $40$. The vertical color bars indicate the mean values of the
corresponding data.
The further to the right around probability $=1 - \delta = 0.3$, the
better the influence guarantee.}
\label{fig:combine}
\end{figure*}

% \begin{figure*}[t]
% \resizebox{.5\textwidth}{!}
% {
% \begin{subfigure}{.45\textwidth}
%   \subimport*{graphs/}{fb.tex}
%   \caption{Facebook dataset.}
% \end{subfigure}
% \begin{subfigure}{.45\textwidth}
%   \subimport*{graphs/}{tw.tex}
%   \caption{Twitter dataset.}
%   \label{fig:ecdf-tw}
% \end{subfigure}
% }
% \begin{subfigure}{.45\textwidth}
%   \subimport*{graphs/}{sd.tex}
%   \caption{Slashdot dataset.}
% \end{subfigure}
% \hfill
% \begin{subfigure}{.45\textwidth}
%   \subimport*{graphs/}{pk.tex}
%   \caption{Pokec dataset.}
% \end{subfigure}
% \caption{
% \small
% Empirical cumulative distribution function (ECDF) of influence of seed sets of size $40$,
% using $100$ samples for each ECDF. The vertical color bars indicate the mean values of the
% corresponding data. For $\infprobheu$ and $\multicritalgo$ algorithms, the seed sets are optimized with
% $\delta=0.7$. Edges activation are random values in the range $[0.001,0.05]$.
% The further to the right around probability $=1 - \delta = 0.3$, the
% better guarantee influence.}
% \label{fig:ecdf}
% \end{figure*}

We examine the empirical performance of our algorithms (\infprobheu{} and \multicritalgo{})
on real world social networks. We study the following questions.\\
%\begin{itemize}
%\item
$\bullet$ How does $\mdelta$ depend on $|S|$ and $\delta$?\\
$\bullet$ How are the actual execution times compared to one another and to the theoretical
sampling sizes? \\
%\item
$\bullet$ How does the performance of our algorithms compare with the solution to the \infmax{}
problem, in terms of both the $\expinf$ and $\mdelta$ objectives? 

%\item
%What are the differences between the characteristics of nodes in the solutions of $\infprob{}$ and $\infmax{}$?
% \item
% What are the approximation ratio and running time of our algorithms in practice?
%\end{itemize}
%
\vspace{-0.05in}
\subsection{Experimental Setup}

\noindent
\textbf{Data sets and tested algorithms}. We use four social network data sets, which were crawled from public sources,
as provided by \cite{snapnets}. The basic statistics of the data sets can be found in Table~\ref{tb:1}.
\begin{figure}
\captionsetup{belowskip=-0.2in}
\resizebox{0.4\textwidth}{!}
{
% \begin{subfigure}{\textwidth}
  \subimport*{graphs/}{delta_pk.tex}
% \end{subfigure}
}
\caption{
\small
Pokec Dataset: Varying $\delta$, with low edge activation in range $[0.001, 0.01]$. For each algorithm, we report the guarantee influence of the output
seed sets of size $15$, $30$, and $40$.
%The lower plots correspond to smaller seed sets.
}
\label{fig:vdelta}
\end{figure}
%

We implement our algorithms for \infprob{} (\infprobheu, and
\multicritalgo), and compare them with the greedy algorithm for \infmax\ (as described in \cite{v011a004}). 
% Since our focus is not on the running time of the algorithm for \infprob{}, 
% we only consider the implementation of \cite{v011a004}, instead of the more recent advances in making it faster, e.g., \cite{borgs2014maximizing,chen2010scalable,tang2014influence,tang2015influence}. 
There have been more recent advances on algorithms to scale the implementation of influence maximization, e.g., \cite{borgs2014maximizing,chen2010scalable,tang2014influence,tang2015influence}; however, these are pretty complicated, and so we use the algorithm of \cite{v011a004} for comparison.
Further, we use $10^4$ samples, following the claim in \cite{v011a004} that these many samples suffice for datasets of this scale, though the worst case bound is  $O(n^2)$ samples, 


For \infprobheu{} and \multicritalgo{}, we take $\eta=0.1$, which is the additive error for the
probability guarantee $\delta$ (as discussed in Section \ref{sec:sampling}).
Following Theorem \ref{thr:numsmpls}, we choose the number of samples to be $100 \times \log(n)$, where $n$ is the number of nodes of the input graph.
%This is a bit lower than the theoretical value to ensure high probability of correctness.
%However, as the experiment results show, the setting is working well in practice.
Finally, recall that \multicritalgo{} can relax the size of the seed set by a multiplicative factor of $\ln(n)$. In order to make the comparison reasonable, we run it with a smaller budget, so that the final seed set size (after the relaxation) is within the bound of $k$; this follows the approach in other works which obtain bicriteria approximation results, e.g., \cite{krause:jmlr08}.

Our codes are implemented in C++ with OpenMP, and executed on a server with $24$ cores and $16GB$ of memory. The source code is published on github.

\noindent
\textbf{Model parameters.} We use the IC model with activation probabilities of $0.01, 0.05, 0.1$, and a two random settings, where the activation probability of each edge is picked uniformly randomly in the range $[0.001, 0.05]$ and $[0.001, 0.01]$.
%As the results show, high activation probability gives trivially high spreading, due to the fact that social graphs are well connected and having small diameters. We then focus on smaller activation: each edge is assigned an activation picked uniformly randomly in the range

For \infprobheu{} and \multicritalgo{}, we conduct two sets of experiments. The first one is configured with probability guarantee $\delta$ in $\{0.2, 0.5, 0.7 0.9\}$.
We observe that the resulted seed sets qualities do not vary much with $\delta$ (consistent with the observation in \cite{zhang:kdd14}), and, therefore, report most of our results for $\delta=0.7$ because of the limited space. In the second set of experiments, we try \infprobheu{} and \multicritalgo{} with extreme values of $\delta$, which are very close to $0$ or $1$.

For the constraint on size of the seed set, $k$, we experiment with all values in $[0,40]$. It is known from the related works that the influence is quickly saturated when one increases $k$, such that increasing the seed set does not have significant gain.
%Combining the parameters setting results in large number of experiments. Thus we will only report the important findings.
%which implies that influence is well concentrated in real social networks. This is the same observation noted in \cite{zhang:kdd14}.

\vspace{-0.05in}
\subsection{Results and Evaluation}
We compare the algorithms by asserting the quality of the
returned seed sets on the original graph, in term of average influence, and empirical cumulative
distribution functions (ECDF).
% Using the \textit{triggering set} technique, we pre-activate the edges to receive a set of fresh samples. Then we calculate the influence of the resulted seed sets
% on these samples. Using that data, we calculate the mean values, and the ECDF's. The ECDF migh not be well estimated, since we do not have any knowledge of the true distributions. The standard approach in this case is to employ non-parametric estimation. We use kernel density estimator (KDE) (\cite{terrell1992variable}, \cite{silverman2018density}) with Gaussian kernel.
% Since all seed sets are measured with the same set of samples, the comparison is fair.
%
\begin{table}
\caption{Data sets.}
\label{tb:1}
\centering
\begin{tabular}{lrrr}
\toprule
Data set & Nodes & Edges & Diameter \\
\midrule
Facebook & 4039 & 88234 & 8 \\
Twitter & 81306 & 1768149 & 7 \\
Slashdot & 77360 & 905468 & 10 \\
Pokec & 1632803 & 30622564 & 11 \\
\bottomrule
\end{tabular}
\end{table}

\smallskip
\noindent
\textbf{Comparison with \expinf{} solution.}
The experiment results with edge activation randomly in the range
$[0.001, 0.05]$, are shown in Figure \ref{fig:combine}.
The seed sets are computed by their
respective algorithms, where both \infprobheu{} and \multicritalgo{} has $\delta$ set to $0.7$.
In Figures \ref{fig:combine-a} and \ref{fig:combine-b}, we observe that
\multicritalgo{} gives the best seed sets, while \infprobheu{} gives comparable or better sets,
versus \infmax{} solutions.

\smallskip
\noindent
\textbf{Execution time.} One of the main advantages of \multicritalgo{} and 
\infprobheu{} algorithms is that their execution times are much smaller compared to the  greedy algorithm for \infmax  (as described in \cite{v011a004}), while the qualities of the solutions obtained are similar or better. In fact, the \multicritalgo{} algorithm is the fastest among the three. The execution time depends on the number of samples. Each sample
complexity depends on the density of the graph, and the edge activation probability. We note that both \multicritalgo{} and 
\infprobheu{} algorithms need significantly less number of samples compared to
that of the algorithm of \cite{v011a004}. We notice that
the variation of $\delta$ do not have much affect on the runtime, and only report the runtime for
$\delta=0.7$, as shown in Table \ref{tb:2}.
% 
\begin{table*}[t]
  \caption{Execution Time, for $k=40$, $\delta=0.7$ (where applicable), and various edge activation
  probabilities. The random probability is uniformly in the range $[0.001, 0.05]$. Runtime is
  measured in seconds. mexp, mprob, and mcrit indicates \infmax, \infprobheu, and \multicritalgo{}
  respectively.}
  \label{tb:2}
  \centering
  \begin{tabular}{lrrrrrrrrrrrr}
  \toprule
  Data set & \multicolumn{3}{c}{rand} & \multicolumn{3}{c}{0.01} & \multicolumn{3}{c}{0.05} & \multicolumn{3}{c}{0.1} \\
  \cmidrule(lr){2-4} \cmidrule(lr){5-7} \cmidrule(lr){8-10} \cmidrule(lr){11-13}
           & mexp & mprob & mcrit & mexp & mprob & mcrit & mexp & mprob & mcrit & mexp & mprob & mcrit \\
  \midrule
  Facebook &   150 &    24 &    \textbf{9} &   138 &    22 &    \textbf{9} &    162 &    24 &    \textbf{9}  &    182 &    26 &   \textbf{10} \\
  Twitter  &  3483 &   690 &  \textbf{242} &  3014 &   634 &  \textbf{249} &   4038 &   799 &  \textbf{252}  &   4216 &   879 &  \textbf{260} \\
  Slashdot &  1906 &   402 &  \textbf{225} &  1825 &   383 &  \textbf{223} &   2130 &   438 &  \textbf{232}  &   2517 &   506 &  \textbf{242} \\
  Pokec    & 59445 & 20067 & \textbf{6553} & 53703 & 16951 & \textbf{6537} &  68646 & 19723 &  \textbf{6850} &  75530 & 22600 & \textbf{6670} \\
  \bottomrule
  \end{tabular}
\end{table*}
% 

%\smallskip
\noindent
\textbf{Probabilistic guarantees.}
We study the probabilistic guarantees of the different algorithms through empirical cumulative
distribution functions (ECDF) \cite{terrell1992variable}.
Due to space constraint, we only report the results for the case where edge activation are picked randomly in the range $[0.001,0.05]$, with $\delta-0.7$. We take the solutions for $k=40$ and fit their
ECDF's by kernel density estimation with Gaussian kernel.
% returned by the algorithms, then evaluate them as follow:
% compute the influence over $100$ fresh samples, and fit their
% ECDF's by kernel density estimation with Gaussian kernel.
These are plotted in figures \ref{fig:combine-c} and \ref{fig:combine-d}, together with
vertical lines indicating the respective mean values. The
measurement $M_{\delta}(.)$ can be asserted by the ECDF at probability $(1-\delta)$. For
example, with $\delta=0.7$, in figure \ref{fig:combine-d}, \multicritalgo{} seed set and \infprobheu{} seed set
is guarantee to have an influence higher than $12100$ for $70$\% of times, which is better than
\infmax{} seed set corresponding value of $10800$.
Generally, the further the ECDF to the right, around probability $(1-\delta)$, the higher the
quality of the seed set as per $M_{\delta}(.)$.

\smallskip
\noindent
\textbf{Effects of varying $\delta$.}
Figure~\ref{fig:vdelta} shows the influence guarantee computed by both algorithms for $\delta\in\{0.005, 0.01, 0.05, 0.1, 0.95, 0.995\}$, with $k\in\{15, 30, 40\}$.
We observe that the $\mdelta$ value from \infprobheu{} decreases steadily with $\delta$, with sharp decrease when $\delta$ is closer to $1$. In contrast, the $\mdelta$ value from  \multicritalgo{} is quite insensitive to $\delta$.
Further, \infprobheu{} has higher $\mdelta$ than \multicritalgo{} for small values of $\delta$.
% We observe that \infprobheu{} is specifically tight around $\delta$ (with a slack of $\eta$ controllable by the number of samples), however without any guarantee. The seed set output by \infprobheu{} is correct as per the probabilistic guarantee, but the influence can be far from optimal.
In particular, \multicritalgo{} gives good solution in most of the scenarios, except when $\delta$ is close to $0$ or $1$, due to the $1/2$ factor in approximating $\delta$.
% This suggests that using \multicritalgo{} for typical values of $\delta$ is the best choice. In special case when we need extreme probabilistic guarantee, the more pessimistic \infprobheu{} might be better.

% The \infprob{} problem has the probability guarantee $\delta$ as an input parameter. We already observed that $\delta$ does not have significant effect on the quality of the optimal seed sets. Which implies in real world graphs, the influence is very well concentrated. Using extreme values of $\delta$ in $\{0.005, 0.01, 0.05, 0.1, 0.95, 0.995\}$, for each data set, we execute \infprobheu{} and \multicritalgo{} to find the optimal seed sets of size $k$ in $\{15, 30, 40\}$, and report the \textit{guarantee influence}.

% \Cref{fig:vdelta} shows how \infprobheu{} outputs sets with lower influence guarantees when $\delta$ is increasing. The decreasing are sharp when $\delta$ is colser to $1$, which is as excpected. \multicritalgo{} is more resistant to the change of $\delta$. This is not to our surprise, since our analysis shows that the approximation is relaxed with $\delta/2$. In all data sets, \infprobheu{} gives higher guarantee influence when $\delta$ is closer to $0$. Except for the Facebook data set, around $\delta=0.4$, \infprobheu{} gives influence guarantees lower than \multicritalgo{} . In the Facebook data set, the smallest one in our experiment, \infprobheu{} outputs decreases earlier, at around $\delta=0.2$.

% The findings confirm the intended characteristics of our algorithms. \infprobheu{} is specifically tight around $\delta$ (with a slack of $\eta$ controllable by the number of samples), however without any guarantee. The seed set output by \infprobheu{} is correct as per the probabilistic guarantee, but the influence can be far from optimal. \multicritalgo{} gives good solution in most of the scenarios, except when $\delta$ is close to $0$ or $1$, due to the $1/2$ factor in approximating $\delta$. We conclude that for real world graphs, where influence tends to be well concentrate, using \multicritalgo{} for typical values of $\delta$ is the best choice. In special case when we need extreme probabilistic guarantee, the more pessimistic \infprobheu{} might be better.
%


\bibliography{refs}

\end{document}
